\subsubsection{Explicación del problema}
Tenemos un plano del cual no nos importan sus dimensiones. Suponiendo que el plano está inicialmente vacío y la entrada del problema consiste en una serie de posiciones que cumplen $X_1 > X_2 > … > X_n \geq 0$ y $0 \leq Y_1 \leq Y_2 \leq … \leq Y_n$, consideramos que esas posiciones en el plano están ocupadas. Además, hay otro parámetro de entrada, $t \geq 0$.
El problema consiste en buscar el menor número de intentos necesarios para remover las posiciones ocupadas. 

Si uno se “ubica” en una posición dada $(x_0, y_0)$, se remueve ese punto y otros, siguiendo una regla: son aquellos cuyas coordenadas $x$ e $y$ cumplen $0 \leq x \leq x_0 + t$, $0 \leq y \leq y_0 + t$.

Por ejemplo, si tenemos sólo dos puntos de entrada, que cumplen las hipótesis, para los cuales sus coordenadas $x$ e $y$ difieren en 2; y el parámetro $t$ = 1. Entonces la cantidad mínima de intentos es 2, ya que si elegimos el punto 1 no se remueve el 2 y viceversa.

\subsubsection{Ideas}

Sabiendo que tienen que desaparecer todos los puntos, en particular tiene que desaparecer el primero de los que aún estan vivos. Entonces, para matarlo intentamos maximizar la efectividad de la genkidama que tiramos, lo que significa que queremos encontrar la genkidama que mata a más enemigos de entre las que lo matan a él.

Para lograr esto hicimos un algoritmo goloso, que lo que hace es:
\begin{itemize}
\item Me paro en el enemigo que aun este vivo, al que llamaremos $E_1$.
\item Avanzo al siguiente a $E_1$, al que llamaremos $E_2$, me fijo si tirarle la genkidama mata a $E_1$.
\item Si lo mata, me fijo iterativamente si el siguiente de $E_2$, llamado $E_3$, al tirarle la genkidama mata a $E_1$... y así sucesivamente hasta llegar al primer enemigo al que tirarle la genkidama no mate a $E_1$ o bien se acaben los enemigos.
\item Si se encontró un enemigo $E_i$ tal que al tirar la genkidama en su posición $E_1$ no muere, entonces tiramos la genkidama en la posición de $E_{i-1}$. Si para todas las posiciones de los enemigos recorridos se mataba a $E_1$, entonces tiramos la genkidama en el último. En este último caso el algoritmo termina, ya no quedan enemigos vivos.
\item Me paro en $E_i$, y me fijo si la genkidama tirada en $E_{i-1}$ lo alcanza. Si lo alcanza, sigo avanzando hasta que no queden enemigos o bien se encuentre uno, $E_j$, para el cual la genkidama tirada en la posición de $E_{i-1}$ no lo mate. Si la genkidama mataba a todos los enemigos restantes, entonces el algoritmo termina.
\item La genkidama mató a todos los enemigos anteriores a $E_j$, entonces ahora éste es el primer enemigo vivo, y se vuelve al principio.

\end{itemize}