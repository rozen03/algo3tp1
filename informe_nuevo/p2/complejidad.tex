\newpage
\subsubsection{Pseudocódigo}
\begin{algorithm}[h]
\caption{Estructura del algoritmo del problema 2}
\begin{algorithmic}[1]
	\Function{Genkidama}{\textbf{vector} Enemigos, \textbf{int} t}
	\State{\textbf{vector} $Soluciones$}
	\State{\textbf{int} $CantSoluciones$ := 0}
	\For{$i$ := 0 hasta Len($Enemigos$)}
		$j$ := $i$ + 1
		\While{Pos($i$) $\leq$ Pos($j$) + $t$}
			\State{$j++$}
		\EndWhile
		\State{$Soluciones$[$cantSoluciones$] := $j$ - 1}
		\State{$CantSoluciones++$}
		\State{$k$ := $j$ + 1}
		\While{Pos($k$) $\leq$ Pos($j$) + $t$}
			\State{$k++$}
		\EndWhile
		$i$ := $k$
	\EndFor
	\State{Print $cantSoluciones$}
	\For{$i$ := 0 hasta $CantSoluciones$}
		\State{Print $Soluciones$[$i$]}
	\EndFor
	\EndFunction
\end{algorithmic}
\end{algorithm}


\subsubsection{Análisis}
%yo creo q esto se entiende, decime vos xD
El algoritmo tiene un ciclo principal con dos ciclos internos. El ciclo principal va recorriendo el vector: el primer ciclo interno recorre los enemigos anteriores al que se elige para tirar la genkidama; el segundo recorre los posteriores que mueren a causa de esta genkidama. Entonces cada enemigo es o bien considerado en el primero o bien en el segundo ciclo, excepto en los casos donde el primer ciclo llega a un enemigo en el cual tirar la genkidama en su posición no mata al primero, entonces vuelve a ser considerado en el segundo ciclo para ver si la genkidama que finalmente se tira en el enemigo anterior lo mata a él.

Para evaluar la complejidad del algoritmo consideramos el mejor, peor y caso promedio.

El mejor caso es el que se produce cuando tiramos la genkidama en cualquiera de los puntos, y esta termina matando a todos los enemigos. En este caso la cantidad de genkidamas mínima es 1 y daría una complejidad de $O(N)$, ya que el primer ciclo va pasando todo el vector hasta llegar al final y el segundo ciclo no tiene ningún caso que considerar. Esto sucede una sola vez, es decir que el ciclo principal solo se ejecuta una vez.

% El peor caso sería si al tirar una genkidama en cualquier enemigo se terminara matando a ese enemigo y sólo a ese enemigo. En este caso cada enemigo termina siendo tomado en cuenta en ambos ciclos internos y en el ciclo principal, dando una complejidad de $O(3N) = O(N)$.

El peor caso sería si al tirar una genkidama en cualquier enemigo se terminara matando a ese enemigo y sólo a ese enemigo. En este caso cada enemigo termina siendo tomado en cuenta en ambos ciclos internos, dando una complejidad de $O(2N) = O(N)$.

% TODO: rozen por lo que puse en una parte de la experimentacion creo que es 2N, lo cambio para que quede igual pero no se que onda.

Y el caso promedio que es simplemente aleatorio también que está en el orden de $O(N)$. Esto es así porque es un caso peor que el mejor y mejor que el peor. Y como el peor y el mejor caso son ambos $O(N)$, no hay otra opción que el caso promedio también lo sea.