El algoritmo se basa en la técnica de backtracking, por lo tanto sin tener en cuenta podas ni complejidad, podemos decir lisa y llanamente que recorremos punto por punto buscando todas las posibles soluciones al problema. Esto es, por cada punto disponible, elegimos otro para trazar una recta y luego verificar cuántos puntos pasan por dicha recta, luego se aplica el mismo procedimiento con los puntos restantes, contando la cantidad de rectas necesarias. Una vez finalizado este proceso volvemos a tomar todos los puntos y nos paramos sobre el primer punto disponible, pero esta vez elegimos otro punto para trazar la recta distinto a la iteración anterior, verificamos cuántos puntos se destruyen y aplicando el mismo procedimiento.
Esto nos asegura evaluar todas las posibles soluciones para todos los puntos disponibles.
En cuanto a las podas, naturalmente verificamos si el punto ya fue destruido y además chequeamos no trazar una recta entre dos puntos en los cuales esos dos puntos sean el mismo, lo cual no sería una recta.
Hasta aquí seguimos analizando todas las posibilidades pues estamos descartando únicamente puntos ya destruidos y no procesar dos veces el mismo punto, por tanto seguimos garantizando el correcto funcionamiento del algoritmo.
Por último, para mejorar el rendimiento y además satisfacer los requerimientos del problema descartamos aquellas soluciones que llegado un momento sean peores que la mejor solución encontrada hasta el momento.
Con las podas correspondientes recién mencionadas podemos asegurar sin problemas que el algoritmo evalúa todas las soluciones factibles a ser la mejor solución y las demás las descarta.

