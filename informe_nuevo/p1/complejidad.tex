\newpage
\subsubsection{Pseudocódigo}
\begin{algorithm}
\caption{Estructura del algoritmo del problema 1}
\begin{algorithmic}[1]
	\Function{KaioKen}{\textbf{int} $N$}
	\State{\textbf{int} $rango$ := $[log_2(N)]$}
	\For{$i$ := $1$ hasta $rango$}
		\For{$x$ := $0$ hasta $N$}
        \If{$x$ mod $2*i < 2*(i-1)$}
            \State{Print “1”}
        \Else
        	\State{Print “2”}
        \EndIf
        \State{$x++$}
        \EndFor
    \State{$i++$} 
	\EndFor
	\EndFunction
\end{algorithmic}
\end{algorithm}
%este deberia ser el post
% Int rango := log2(n) + 2
% If (n == 2(rango -2 )) rango := rango - 1
% Para i := 1 hasta rango hacer
% 	Para x := 0 hasta n hacer
% 		Si x mod 2i < 2(i – 1) entonces 
% Imprimir “1”
% 		Else aw
% Imprimir “2”
% 		Fin Si
% 		X++
% 	Fin Para
% 	I++
% Fin Para

\subsubsection{Análisis} 
%EL corrector dijo que esto estaba bien, si demostraba bien la correctitud, pero no se si hace falta mas  explicacion
El algoritmo hace dos ciclos anidados. El primero lo hace $log_2(N)$ veces y el otro $N$ veces, mientras que todas las operaciones adentro de ambos ciclos tienen costo de $O(1)$ dando una complejidad final de $O(NLog(N))$.