\subsubsection{Pseudocódigo}
\begin{algorithm}[h!]
\caption{Estructura del algoritmo del problema 1}
\begin{algorithmic}[1]
	\Function{KaioKen}{\textbf{int} $N$}
	\State{\textbf{int} $rango$ := $[log_2(N)]$}
	\For{$i$ := $1$ hasta $rango$}
		\For{$x$ := $0$ hasta $N$}
        \If{$x$ mod $2*i < 2*(i-1)$}
            \State{Print “1”}
        \Else
        	\State{Print “2”}
        \EndIf
        \State{$x++$}
        \EndFor
    \State{$i++$} 
	\EndFor
	\EndFunction
\end{algorithmic}
\end{algorithm}

\subsubsection{Análisis} 
%EL corrector dijo que esto estaba bien, si demostraba bien la correctitud, pero no se si hace falta mas  explicacion
El algoritmo hace dos ciclos anidados. El primero lo hace $log_2(N)$ veces y el otro $N$ veces, mientras que todas las operaciones adentro de ambos ciclos tienen costo de $O(1)$ dando una complejidad final de $O(NLog(N))$.