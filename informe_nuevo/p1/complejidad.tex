\subsubsection{Pseudocódigo}
\begin{algorithm}[h!]
\caption{Estructura del algoritmo de D\&C}
\begin{algorithmic}[1]
	\Function{D\&C}{}
	\State {defino la cantidad de instancias como la parte entera de $log_2(n)+2$}
	\If{$N$ es potencia de $2^k$}
		\State le resto 1 a la cantidad de instancias 
	\EndIf
	\For{k entre 1 y la cantidad de instancias}
		\For{cada indice de peleador}
        \If{el resto del indice de $2^k$ es menor a $2^{k-1}$}
            \State el peleador va al primer bando
        \Else
        	\State el peleador va al segundo bando
        \EndIf
        \EndFor 
	\EndFor
	\EndFunction
\end{algorithmic}
\end{algorithm}
%este deberia ser el post
% Int rango := log2(n) + 2
% If (n == 2(rango -2 )) rango := rango - 1
% Para i := 1 hasta rango hacer
% 	Para x := 0 hasta n hacer
% 		Si x mod 2i < 2(i – 1) entonces 
% Imprimir “1”
% 		Else aw
% Imprimir “2”
% 		Fin Si
% 		X++
% 	Fin Para
% 	I++
% Fin Para

\subsubsection{Analisis} 
%EL corrector dijo que esto estaba bien, si demostraba bien la correctitud, pero no se si hace falta mas  explicacion
El algoritmo hace dos ciclos anidados el primero que lo hace $Log_2(N)$ veces y el otro $N$ veces, mientras que todas las operaciones adentro de ambos ciclos tiene costo de $O(1)$ dando una complejidad final de $O(N Log(N))$.