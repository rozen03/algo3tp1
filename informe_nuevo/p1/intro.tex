\subsubsection{Explicación del problema}
\noindent Tenemos N elementos (peleadores) y el problema consiste en dividir esos elementos en 2 grupos, generando diferentes instancias; cumpliendo con el siguiente criterio de división:
\begin{itemize}
\item los elementos tienen que estar repartidos en tantas instancias como se requieran para que cada uno esté “enfrentado” a cada otro elemento al menos una vez considerando todas las instancias de salida que arroja el problema
\item la cantidad de instancias generadas debe ser mínima
\item en cada una de dichas instancias deben participar los N elementos de entrada
\end{itemize}
Con “enfrentarse” nos referimos a que si tenemos únicamente 2 elementos, A y B, la manera de “enfrentarlos” va a ser repartiendo en un grupo el elemento A y en otro grupo el elemento B.
Por ejemplo, si N = 3, llamando A, B y C a los elementos: 

Para resolver el problema podemos enfrentar a A con B y C, de esta manera nos queda únicamente enfrentar a B con C. Podemos entonces dividir los 3 elementos en una segunda instancia pero esta vez con B por un lado y C por el otro. Entonces, hay dos soluciones posibles: o bien ponemos a C solo o bien a B solo.

\subsubsection{Idea e implementación}
\noindent Para resolver el problema usaremos la técnica de Divide \& Conquer:

En el modelo del problema representamos a cada peleador con un número natural. El número de instancias a ser generadas es $[log_{2}(N)]$, donde $N$ es la cantidad de peleadores. Para cada instancia correspondiente a $1 \leq k \leq N$:
\begin{description}
\item[Didivir] - Separamos a los peleadores segun el resto de su índice respecto a $2^k$.
\item[Conquistar] - Agrupamos a los peleadores en 2 bandos: uno conteniendo a los peleadores cuyo resto es menor a $2^{k-1}$, y otro con los restantes
\end{description}

