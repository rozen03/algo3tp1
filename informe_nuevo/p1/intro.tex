\subsubsection{Explicación del problema}
\noindent Tenemos N elementos (peleadores) y el problema consiste en dividir esos elementos en 2 grupos, generando diferentes instancias, cumpliendo con el siguiente criterio de división:
\begin{itemize}
\item los elementos tienen que estar repartidos en tantas instancias como se requieran para que cada uno esté “enfrentado” a cada otro elemento al menos una vez considerando todas las instancias de salida que arroja el problema
\item la cantidad de instancias generadas debe ser mínima
\item en cada una de dichas instancias deben participar los N elementos de entrada
\end{itemize}
Con “enfrentarse” nos referimos a que si tenemos únicamente 2 elementos, A y B, la manera de “enfrentarlos” va a ser repartiendo en un grupo el elemento A y en otro grupo el elemento B.
Por ejemplo, si N = 3, llamando A, B y C a los elementos: 

Para resolver el problema podemos enfrentar a A con B y C, de esta manera nos queda únicamente enfrentar a B con C. Podemos entonces dividir los 3 elementos en una segunda instancia pero esta vez con B por un lado; y A y C por el otro. Cabe destacar que también podemos ubicar a C por un lado; y A y B por otro, y también cumpliríamos con la resolución, pero no sería minimal que es lo que pide el problema, en ese caso o bien optamos por poner a B solo o bien optamos por poner a C solo.
\subsubsection{Idea e implementacion}
.\\
Para resolver el problema usaremos la tecnica de Divide \& Conquer: \\
	Didivir: Instancia separamos a los peleadores segun el resto de su indice respecto a $2^k$, siendo $k$ la instancia \\
	Conquistar: Los agrupamos a los peleadores en 2 bandos, uno que agrupo a los luchadores segun el indice que cuyo resto sea menor a $2^k$ y el otro a los restantes. \\

 En la modelacion del problema, representamos cada peleador con un indice natural , y los separamos en el bando 1 o 2 Log(N) veces con N como la cantidad de peleadores.
% no se si esto explica mejor
% El criterio para separarlos es que los peleadores es:
% tomo el indice del peleador
% tomo el resto de dividir el indice por 2^k, donde k es el numero de instancia que estamos creando
% si el resto es menor a 2^{k-1} el peleador va al  grupo 1 sino el peleador va al grupo 2

