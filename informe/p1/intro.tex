\subsubsection{Explicación del problema}
El problema consiste en armar varias instancias de 2 bandos de peleadores de forma tal que, siempre participen todos los peleadores en cada instancia, cada peleador haya peleado contra otro al menos una vez y que la cantidad de estas instancias sea minima.
Para resolver este problema, modelamos cada peleador con un indice, es decir numero natural.
\subsubsection{Idea e implementacion}
Para resolver el problema usaremos la tecnica de Divide \& Conquer:
-Didivir: Instancia separamos a los peleadores segun el resto de su indice respecto a $2^k$, siendo $k$ la instancia
-Conquista: Los agrupamos a los peleadores en 2 bandos, uno que agrupo a los luchadores segun el indice que cuyo resto sea menor a $2^k$ y el otro a los restantes.

De forma tal que siempre logramos que un peleador llegue a enfrentado a cualquier otro para cuando llegamos a la cota de $k = log_2 (N)$ siendo $N$ la cantidad de peleadores


\begin{algorithm}[h!]
\caption{Estructura del algoritmo de D\&C}
\begin{algorithmic}[1]
	\Function{D\&C}{}
	\state {defino la cantidad de instancias como la parte entera de $log_2(n)+2$}
	\If{$N$ es potencia de $2^k$}
		\State le resto 1 a la cantidad de instancias 
	\EndIf
	\For{k entre 1 y la cantidad de instancias}
		\For{cada indice de peleador}
        \If{el resto del indice de $2^k$ es menor a $2^{k-1}$}
            \State el peleador va al primer bando
        \Else
        	\State el peleador va al segundo bando
        \EndIf
        \EndFor
	\EndFor
	\EndFunction
\end{algorithmic}
\end{algorithm}
