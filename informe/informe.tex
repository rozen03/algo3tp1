\documentclass[a4paper, 12pt, spanish]{article}

\usepackage[paper=a4paper, left=1.5cm, right=1.5cm, bottom=1.5cm, top=3.5cm]{geometry}
\usepackage[spanish, es-noshorthands]{babel}
\usepackage[utf8]{inputenc}
\usepackage[none]{hyphenat}
\usepackage[colorlinks,citecolor=black,filecolor=black,linkcolor=black,    urlcolor=black]{hyperref}

% Simbolos matemáticos
\usepackage{amsmath}
\usepackage{amsfonts}
\usepackage{amssymb}
\usepackage{algorithm}
\usepackage[noend]{algpseudocode}
\usepackage{algorithmicx}
\usepackage{listings}

% Descoración y gráficos
\usepackage{caratula}
\usepackage{graphicx} 
\usepackage{fancyhdr}
\usepackage{lastpage}
\usepackage{caption}
\usepackage{subcaption}
\usepackage{multirow}
\usepackage{alltt}
\usepackage{tikz}
\usepackage{color}
\usepackage{gnuplottex}

% Acomodo fancyhdr.
\pagestyle{fancy}
\thispagestyle{fancy}
\addtolength{\headheight}{1pt}
\lhead{Algoritmos y Estructuras de Datos III}
\rhead{$1^{\mathrm{do}}$ cuatrimestre de 2016}
\cfoot{\thepage /\pageref*{LastPage}}
\renewcommand{\footrulewidth}{0.4pt}

\floatname{algorithm}{Pseudocódigo}
\algrenewcommand\algorithmicfunction{\textbf{Función}}
\algrenewcommand\algorithmicfor{\textbf{para}}
\algrenewcommand\algorithmicdo{\textbf{hacer:}}
\algrenewcommand\algorithmicif{\textbf{si}}
\algrenewcommand\algorithmicthen{\textbf{entonces:}}
\algrenewcommand\algorithmicelse{\textbf{si no:}}
\algrenewcommand\algorithmicend{\textbf{fin}}
\algrenewcommand\algorithmicreturn{\textbf{devolver}}

\sloppy

\parskip=5pt % 10pt es el tama de fuente

% Pongo en 0 la distancia extra entre itemes.
\let\olditemize\itemize
\def\itemize{\olditemize\itemsep=0pt}


\materia{Algoritmos y Estructuras de Datos III}
\grupo{Conformación del grupo}
\tituloCaratula{Trabajo Práctico N$^\circ$1}

\integrante{Fernando Abelini}{544/09}{ferabelini@outlook.com}
\integrante{Jose Fernando Alvaro}{89/10}{fer1578@gmail.com}
\integrante{Uriel Jonathan Rozenberg}{838/12}{urielrozenberg@hotmail.com}
\integrante{Roman Litvin}{183/14}{rglitvin@gmail.com}

\usepackage{tikz}
%\usepackage{tikz-qtree}


\usetikzlibrary{arrows,backgrounds,calc}

\pgfdeclarelayer{background}
\pgfsetlayers{background,main}



\newcommand{\convexpath}[2]{
[ 
    create hullnodes/.code={
        \global\edef\namelist{#1}
        \foreach [count=\counter] \nodename in \namelist {
            \global\edef\numberofnodes{\counter}
            \node at (\nodename) [draw=none,name=hullnode\counter] {};
        }
        \node at (hullnode\numberofnodes) [name=hullnode0,draw=none] {};
        \pgfmathtruncatemacro\lastnumber{\numberofnodes+1}
        \node at (hullnode1) [name=hullnode\lastnumber,draw=none] {};
    },
    create hullnodes
]
($(hullnode1)!#2!-90:(hullnode0)$)
\foreach [
    evaluate=\currentnode as \previousnode using \currentnode-1,
    evaluate=\currentnode as \nextnode using \currentnode+1
    ] \currentnode in {1,...,\numberofnodes} {
-- ($(hullnode\currentnode)!#2!-90:(hullnode\previousnode)$)
  let \p1 = ($(hullnode\currentnode)!#2!-90:(hullnode\previousnode) - (hullnode\currentnode)$),
    \n1 = {atan2(\x1,\y1)},
    \p2 = ($(hullnode\currentnode)!#2!90:(hullnode\nextnode) - (hullnode\currentnode)$),
    \n2 = {atan2(\x2,\y2)},
    \n{delta} = {-Mod(\n1-\n2,360)}
  in 
    {arc [start angle=\n1, delta angle=\n{delta}, radius=#2]}
}
-- cycle
}

\newcommand{\todo}[1]{
\textbf{\color{red}{\underline{Nota:} #1}}
}

\newcommand\param[3]{\ensuremath{\mathbf{\textbf{#1}}\,#2\!:} \texttt{#3}}

\let\state\State
\let\while\While
\let\endwhile\EndWhile
\let\endif\EndIf
\let\elseif\ElsIf
\let\for\For
\let\endfor\EndFor
\let\function\Function
\let\endfunction\EndFunction


\begin{document}

%\setcounter{tocdepth}{2}

\begin{titlepage}

\maketitle

\end{titlepage}

\tableofcontents

\newpage
\section{Problema 1:  Kaio Ken}
\subsection{Introducción}
\subsubsection{Explicación del problema}
El problema consiste 
% Tenemos un plano del cual no nos importan sus dimensiones. Suponiendo que el plano en estado inicial está vacío y la entrada del problema consiste en una serie posiciones que cumplen que X1 > X2 > … > Xn >= 0 y 0 <= Y1 <= Y2 <= … <= Yn, entonces podemos considerar que esas posiciones en el plano están ocupadas.
% El problema consiste en buscar el menor número de intentos necesarios para remover las posiciones ocupadas. 
% Consideramos que un punto se remueve si uno se “ubica” en esa posición y considera que se remueve ese punto y todos los demás puntos que están a una distancia T sobre ambos ejes, siendo T una entrada del programa que es un número entero positivo.
% Supongamos N = 2, tenemos una grilla de 2x2, y suponiendo que tenemos los puntos ubicados de la siguiente manera:

% Si T = 0 entonces el menor número de intentos que necesito para remover ambos puntos es 2. Si T fuera mayor o igual a 1 entonces con un solo intento ya removimos ambos puntos.
\subsubsection{Ideas}
La idea de la solucion era buscar una forma en la cual poder definir grpos de enemigos para averiguar en cual convenia tirar la genkidama para que sea la menor cantidad.
Para eso lo que decidimos fue hacer un algoritmo goloso que consiste en tomar el primero del conjunto e ir recorriendo los enemigos en orden hasta llegar al ultimo de los que terminarian haciendo que muera el primero le tiro una genkidama. Nos guardamos a ese enemigo y luego evaluamos hasta que enemigo llegaria la genkidama si lo tiramos en el, y volvemos a repetir el proceso a partir del siguiente enemigo hasta finalizar la lista de enemigos.

%supongo q lo q dice despues de "algoritmo goloso" deberia eliminarse y pegar esto
%  Me paro en el enemigo que aun este vivo, al que llamaremos E1.
% Me paro en el siguiente a E1, al que llamaremos E2, me fijo si tirarle la genkidama mata a E1.
% si lo mata, me fijo iterativamente si el siguiente de E2, llamado E3, me fijo si tirarle la genkidama mata a E1.... y asi susesivamente hasta llegar al ultimo enemigo que tirarle la genkidama mate a E1 o bien se acaben los enemigos.
% para el ultimo enemigo marcado, lo llamamos Ei.
% Decinimos que Ei sera donde se tirara la siguiente genkidama.
% Me paro en el siguiente enemigo de Ei, lo llamamos Ei+1, y nos fijamos si, tirando la genkidama en Ei, muere tambien E1+1.
% luego repetimos el proceso con Ei+2 y asi sucesitvamente hasta llegar al ultimo enemigo que muera al tirar la genkidama en Ei o bien al ultimo enemigo restante y lo llamamos Ej.

% y vuelvo al principio si aun quedan enemigos vivos

\subsection{Correctitud}
Para demostrar la correctitud del algoritmo vamos a probarlo en 3 partes.
1.Probar que para todo par de naturales existe un k tal que sus restos son diferentes
2.Probar que el minimo k posible para cualqueir conjunto es log(n) 
3.probar que entonces para cada par naturales en alguna instancial van a estar en un bando o en otro, 


Prologo: En la modelacion del problema, representamos cada peleador con un indice natural , y los separamos en el bando 1 o 2 Log(N) veces con N como la cantidad de peleadores.
El criterio para separarlos es que los peleadores es:
tomo el indice del peleador
tomo el resto de dividir el indice por 2^k, donde k es el numero de instancia que estamos creando
si el resto es menor a 2^{k-1} el peleador va al  grupo 1 sino el peleador va al grupo 2

IDEA:
1.Basta con tomar K tal que 2^k >= N =>2^{k-1}, entonces el resto de cada par i,j \in [1,_,N] su resto va a ser i y j respectivamente, entonces son diferentes
2.Ese k elegido es obligatoriamente k = log_2(N) por propiedades aritmeticas, ahora, yo se que hasta k-1 existe un i tal que el resto de N de 2^x es igual al resto de i de 2^x para todo x entre 0 y k-1,(hace falta probarlo? es trivialmente aritmetico) entonces para lograr que esten en otro grupo
3.Probar esto tenemos que probar que para todo par de naturales  i,j existe un k tal que:
	El resto de dividir i por 2^k es mayor a 2^{k-1} y El resto de dividir j por 2^k es menor a 2^{k-1}  
basta con pensar la representacion en binario de i y j, el resto de 2^k es tomar desde el bit 0 hasta el kesimo bit, cuestion de que si tenemos el kesimo bit de cada numero, separariamos entre los q el bit k-1 es 1 y los que el bit es 0,siendo i_k el kaesimo bit, es cuestion de probar por absurdo que  i_k =1 y j_k=0 para algun k, usando que si todos sus coeficient es son unos o ceros en los mismos indices, entonces es el mismo numero, que eso era absurdo por la hipotesis, entonces, siempre va pasar q haya una instancia en la que 
\subsection{Complejidad}
\newpage
\subsubsection{Pseudocódigo}
\begin{algorithm}[h]
\caption{Estructura del algoritmo del problema 2}
\begin{algorithmic}[1]
	\Function{Genkidama}{\textbf{vector} Enemigos, \textbf{int} t}
	\State{\textbf{vector} $Soluciones$}
	\State{\textbf{int} $CantSoluciones$ := 0}
	\For{$i$ := 0 hasta Len($Enemigos$)}
		$j$ := $i$ + 1
		\While{Pos($i$) $\leq$ Pos($j$) + $t$}
			\State{$j++$}
		\EndWhile
		\State{$Soluciones$[$cantSoluciones$] := $j$ - 1}
		\State{$CantSoluciones++$}
		\State{$k$ := $j$ + 1}
		\While{Pos($k$) $\leq$ Pos($j$) + $t$}
			\State{$k++$}
		\EndWhile
		$i$ := $k$
	\EndFor
	\State{Print $cantSoluciones$}
	\For{$i$ := 0 hasta $CantSoluciones$}
		\State{Print $Soluciones$[$i$]}
	\EndFor
	\EndFunction
\end{algorithmic}
\end{algorithm}


\subsubsection{Análisis}
%yo creo q esto se entiende, decime vos xD
El algoritmo tiene un ciclo principal con dos ciclos internos. El ciclo principal va recorriendo el vector: el primer ciclo interno recorre los enemigos anteriores al que se elige para tirar la genkidama; el segundo recorre los posteriores que mueren a causa de esta genkidama. Entonces cada enemigo es o bien considerado en el primero o bien en el segundo ciclo, excepto en los casos donde el primer ciclo llega a un enemigo en el cual tirar la genkidama en su posición no mata al primero, entonces vuelve a ser considerado en el segundo ciclo para ver si la genkidama que finalmente se tira en el enemigo anterior lo mata a él.

Para evaluar la complejidad del algoritmo consideramos el mejor, peor y caso promedio.

El mejor caso es el que se produce cuando tiramos la genkidama en cualquiera de los puntos, y esta termina matando a todos los enemigos. En este caso la cantidad de genkidamas mínima es 1 y daría una complejidad de $O(N)$, ya que el primer ciclo va pasando todo el vector hasta llegar al final y el segundo ciclo no tiene ningún caso que considerar. Esto sucede una sola vez, es decir que el ciclo principal solo se ejecuta una vez.

El peor caso sería si al tirar una genkidama en cualquier enemigo se terminara matando a ese enemigo y sólo a ese enemigo. En este caso cada enemigo termina siendo tomado en cuenta en ambos ciclos internos y en el ciclo principal, dando una complejidad de $O(3N) = O(N)$.

Y el caso promedio que es simplemente aleatorio también que está en el orden de $O(N)$. Esto es así porque es un caso peor que el mejor y mejor que el peor. Y como el peor y el mejor caso son ambos $O(N)$, no hay otra opción que el caso promedio también lo sea.
\subsection{Analisis experimental}

% Para el análisis consideramos necesario experimentar únicamente aumentando el valor de entrada, ya que no tenemos ni mejor caso ni peor caso porque el algoritmo sin excepciones recorre todo el vector una cantidad logarítmica de veces según la entrada N, con lo cual presentamos un gráfico mostrando el valor de la entrada aumentando en potencias de 2. Con lo cual se deduce que, cuanto mayor es el tamaño de la entrada, más ciclos de CPU son necesarios y por tanto más tiempo.
% Y ademas, se observa que, con ayuda de la funcion de ajuste, el algortimo tiene comportamiento  de nlogn


\newpage
\input{Problema2}
\newpage
\section{Problema 3: Kamehameha}
\subsection{Introducción}
\subsubsection{Explicación del problema}
El problema consiste 
% Tenemos un plano del cual no nos importan sus dimensiones. Suponiendo que el plano en estado inicial está vacío y la entrada del problema consiste en una serie posiciones que cumplen que X1 > X2 > … > Xn >= 0 y 0 <= Y1 <= Y2 <= … <= Yn, entonces podemos considerar que esas posiciones en el plano están ocupadas.
% El problema consiste en buscar el menor número de intentos necesarios para remover las posiciones ocupadas. 
% Consideramos que un punto se remueve si uno se “ubica” en esa posición y considera que se remueve ese punto y todos los demás puntos que están a una distancia T sobre ambos ejes, siendo T una entrada del programa que es un número entero positivo.
% Supongamos N = 2, tenemos una grilla de 2x2, y suponiendo que tenemos los puntos ubicados de la siguiente manera:

% Si T = 0 entonces el menor número de intentos que necesito para remover ambos puntos es 2. Si T fuera mayor o igual a 1 entonces con un solo intento ya removimos ambos puntos.
\subsubsection{Ideas}
La idea de la solucion era buscar una forma en la cual poder definir grpos de enemigos para averiguar en cual convenia tirar la genkidama para que sea la menor cantidad.
Para eso lo que decidimos fue hacer un algoritmo goloso que consiste en tomar el primero del conjunto e ir recorriendo los enemigos en orden hasta llegar al ultimo de los que terminarian haciendo que muera el primero le tiro una genkidama. Nos guardamos a ese enemigo y luego evaluamos hasta que enemigo llegaria la genkidama si lo tiramos en el, y volvemos a repetir el proceso a partir del siguiente enemigo hasta finalizar la lista de enemigos.

%supongo q lo q dice despues de "algoritmo goloso" deberia eliminarse y pegar esto
%  Me paro en el enemigo que aun este vivo, al que llamaremos E1.
% Me paro en el siguiente a E1, al que llamaremos E2, me fijo si tirarle la genkidama mata a E1.
% si lo mata, me fijo iterativamente si el siguiente de E2, llamado E3, me fijo si tirarle la genkidama mata a E1.... y asi susesivamente hasta llegar al ultimo enemigo que tirarle la genkidama mate a E1 o bien se acaben los enemigos.
% para el ultimo enemigo marcado, lo llamamos Ei.
% Decinimos que Ei sera donde se tirara la siguiente genkidama.
% Me paro en el siguiente enemigo de Ei, lo llamamos Ei+1, y nos fijamos si, tirando la genkidama en Ei, muere tambien E1+1.
% luego repetimos el proceso con Ei+2 y asi sucesitvamente hasta llegar al ultimo enemigo que muera al tirar la genkidama en Ei o bien al ultimo enemigo restante y lo llamamos Ej.

% y vuelvo al principio si aun quedan enemigos vivos

\subsection{Correctitud}
Para demostrar la correctitud del algoritmo vamos a probarlo en 3 partes.
1.Probar que para todo par de naturales existe un k tal que sus restos son diferentes
2.Probar que el minimo k posible para cualqueir conjunto es log(n) 
3.probar que entonces para cada par naturales en alguna instancial van a estar en un bando o en otro, 


Prologo: En la modelacion del problema, representamos cada peleador con un indice natural , y los separamos en el bando 1 o 2 Log(N) veces con N como la cantidad de peleadores.
El criterio para separarlos es que los peleadores es:
tomo el indice del peleador
tomo el resto de dividir el indice por 2^k, donde k es el numero de instancia que estamos creando
si el resto es menor a 2^{k-1} el peleador va al  grupo 1 sino el peleador va al grupo 2

IDEA:
1.Basta con tomar K tal que 2^k >= N =>2^{k-1}, entonces el resto de cada par i,j \in [1,_,N] su resto va a ser i y j respectivamente, entonces son diferentes
2.Ese k elegido es obligatoriamente k = log_2(N) por propiedades aritmeticas, ahora, yo se que hasta k-1 existe un i tal que el resto de N de 2^x es igual al resto de i de 2^x para todo x entre 0 y k-1,(hace falta probarlo? es trivialmente aritmetico) entonces para lograr que esten en otro grupo
3.Probar esto tenemos que probar que para todo par de naturales  i,j existe un k tal que:
	El resto de dividir i por 2^k es mayor a 2^{k-1} y El resto de dividir j por 2^k es menor a 2^{k-1}  
basta con pensar la representacion en binario de i y j, el resto de 2^k es tomar desde el bit 0 hasta el kesimo bit, cuestion de que si tenemos el kesimo bit de cada numero, separariamos entre los q el bit k-1 es 1 y los que el bit es 0,siendo i_k el kaesimo bit, es cuestion de probar por absurdo que  i_k =1 y j_k=0 para algun k, usando que si todos sus coeficient es son unos o ceros en los mismos indices, entonces es el mismo numero, que eso era absurdo por la hipotesis, entonces, siempre va pasar q haya una instancia en la que 
\subsection{Complejidad}
\newpage
\subsubsection{Pseudocódigo}
\begin{algorithm}[h]
\caption{Estructura del algoritmo del problema 2}
\begin{algorithmic}[1]
	\Function{Genkidama}{\textbf{vector} Enemigos, \textbf{int} t}
	\State{\textbf{vector} $Soluciones$}
	\State{\textbf{int} $CantSoluciones$ := 0}
	\For{$i$ := 0 hasta Len($Enemigos$)}
		$j$ := $i$ + 1
		\While{Pos($i$) $\leq$ Pos($j$) + $t$}
			\State{$j++$}
		\EndWhile
		\State{$Soluciones$[$cantSoluciones$] := $j$ - 1}
		\State{$CantSoluciones++$}
		\State{$k$ := $j$ + 1}
		\While{Pos($k$) $\leq$ Pos($j$) + $t$}
			\State{$k++$}
		\EndWhile
		$i$ := $k$
	\EndFor
	\State{Print $cantSoluciones$}
	\For{$i$ := 0 hasta $CantSoluciones$}
		\State{Print $Soluciones$[$i$]}
	\EndFor
	\EndFunction
\end{algorithmic}
\end{algorithm}


\subsubsection{Análisis}
%yo creo q esto se entiende, decime vos xD
El algoritmo tiene un ciclo principal con dos ciclos internos. El ciclo principal va recorriendo el vector: el primer ciclo interno recorre los enemigos anteriores al que se elige para tirar la genkidama; el segundo recorre los posteriores que mueren a causa de esta genkidama. Entonces cada enemigo es o bien considerado en el primero o bien en el segundo ciclo, excepto en los casos donde el primer ciclo llega a un enemigo en el cual tirar la genkidama en su posición no mata al primero, entonces vuelve a ser considerado en el segundo ciclo para ver si la genkidama que finalmente se tira en el enemigo anterior lo mata a él.

Para evaluar la complejidad del algoritmo consideramos el mejor, peor y caso promedio.

El mejor caso es el que se produce cuando tiramos la genkidama en cualquiera de los puntos, y esta termina matando a todos los enemigos. En este caso la cantidad de genkidamas mínima es 1 y daría una complejidad de $O(N)$, ya que el primer ciclo va pasando todo el vector hasta llegar al final y el segundo ciclo no tiene ningún caso que considerar. Esto sucede una sola vez, es decir que el ciclo principal solo se ejecuta una vez.

El peor caso sería si al tirar una genkidama en cualquier enemigo se terminara matando a ese enemigo y sólo a ese enemigo. En este caso cada enemigo termina siendo tomado en cuenta en ambos ciclos internos y en el ciclo principal, dando una complejidad de $O(3N) = O(N)$.

Y el caso promedio que es simplemente aleatorio también que está en el orden de $O(N)$. Esto es así porque es un caso peor que el mejor y mejor que el peor. Y como el peor y el mejor caso son ambos $O(N)$, no hay otra opción que el caso promedio también lo sea.
\subsection{Analisis experimental}

% Para el análisis consideramos necesario experimentar únicamente aumentando el valor de entrada, ya que no tenemos ni mejor caso ni peor caso porque el algoritmo sin excepciones recorre todo el vector una cantidad logarítmica de veces según la entrada N, con lo cual presentamos un gráfico mostrando el valor de la entrada aumentando en potencias de 2. Con lo cual se deduce que, cuanto mayor es el tamaño de la entrada, más ciclos de CPU son necesarios y por tanto más tiempo.
% Y ademas, se observa que, con ayuda de la funcion de ajuste, el algortimo tiene comportamiento  de nlogn



\ref{LastPage}

\end{document}
